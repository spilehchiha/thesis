% !TEX root = ../../mythesis.tex

\chapter{Background}
\label{chap:background}
    This chapter reviews the necessary background knowledge for the reader to be better acquainted with the work conducted within this thesis.
    It highlights the foundational technical basics of Ethereum blockchain, smart contracts, and the most common vulnerabilities associated with smart contracts.
    We cover these technicalities in the following order (based on the work of~\cite{ferreira2022smart}):
        First, we go over the basics of Ethereum, components, and structure. We will go through how blocks are formed,
        what sort of accounts exist on the Ethereum network, and how transactions are executed.
        We will also go through how the Ethereum Virtual Machine functions.
        Afterward, we will go over smart contracts and their most common vulnerabilities out in the wild.
        We will discuss Solidity-written source code and bytecode of smart contracts and explain each vulnerability according to the DASP 10 classification.~\cite{dasp}


\section{Ethereum}
    Ethereum is a decentralized virtual machine introduced as alternative blockchain technology to Bitcoin in 2014 by~\cite{wood2014ethereum}.
    A blockchain is a peer-to-peer network made up of computers that act as nodes and distribute updates for a single database without necessarily having confidence in one another.
    It is based on a combination of cryptography, networking, and incentive mechanisms.~\cite{wohrer2018smart}
    The database mentioned above effectively serves as a ledger, recording every transaction that each node in the blockchain network makes. 


    \begin{figure}
        \centering
        \includegraphics[width=\textwidth]{figures/ethereum-blocks.png}
        \caption{Ethereum blockchain structure.}
        \label{fig:ethereumBlockchainStructure}
    \end{figure}

    As the second most popular blockchain, the Ethereum blockchain was developed as an alternative to cover the lackings of Bitcoin.
    It is a transaction-based, cryptographically secure state machine that reads a series of inputs and, based on those inputs, transitions to a new state.~\cite{ferreira2022smart}
    Like Bitcoin, Ethereum currently uses Proof-of-Work (PoW) as its consensus protocol.
    Proof-of-Stake (PoS), PBFT(Practical Byzantine Fault Tolerance), and DPoS (Delegated Proof of Stake) are other forms of a consensus protocol, for example.
    The solution to a series of cryptographic puzzles is used in the PoW mechanism to prove the credibility of the data being written on the blockchain, using such mechanism as their consensus protocol.
    The puzzle is usually a computationally hard but easily verifiable mathematical problem.
    When a node creates a block, it must resolve a PoW puzzle and spend computing power to achieve so. The nodes compete with each other over this objective function, and the node with the most computing power usually succeeds.
    After the PoW puzzle is resolved, it will be broadcasted to other nodes to achieve the purpose of consensus and append a new block to the blockchain.
    It is "proof" that a node has done "work" by spending its computational resources.
    This process is known as mining, and nodes that decide to participate in this process and try to create new blocks are known as miners.

    \begin{figure}
        \centering
        \includegraphics[width=\textwidth]{figures/uncle.png}
        \caption{An illustration of Ethereum's GHOST protocol.}
        \label{fig:uncle}
    \end{figure}

    \subsection{Accounts}
        The Ethereum state consists of many small objects named accounts, where each account has a 20-byte address and state transitions to interact with the other accounts on-chain.
        An address on the Ethereum blockchain is a 160-bit identifier used to identify any account.
        The world state is a mapping between addresses and account states.~\cite{wood2014ethereum}
        Ethereum supports two types of accounts: externally owned
        (controlled by private keys) -namely EOA's- and contract accounts (CA'S, controlled by their contract code)~\cite{ethereum2014ethereum}.
        Inside of an Ethereum account is composed of four fields: nonce, ether balance, contract code hash, and storage root, explained as follows:

        \begin{itemize}
            \item \textbf{Nonce:} Nonce represents the number of transactions sent from a particular address or the number of contract creations made by an account and is used to guarantee that each transaction can only be processed once.
            \item \textbf{Balance:} Ether balance is the number of Wei owned by this address
                Wei is the smallest subunit of Ether (1 Wei is equivalent to 10-18 ether).
            \item \textbf{Storage Root:} Storage root is the 256-bit hash of the root node of a Merkle Patricia tree that represents the content of the account 
            \item \textbf{Contract CodeHash:} Contract code hash is the Keccak-256 hash of the Ethereum Virtual Machine (EVM) code of the account, which is executed if an address receives a message call.
        \end{itemize}

    \subsection{Transactions}
    A transaction is a cryptographically signed instruction sent by an account on the network towards another.
    There exist only two types of transactions based on the outcomes they generate:
    \begin{itemize}
        \item Message calls, which are created by contract accounts to produce and execute a message that leads to the recipient account (an EOA or contract account) running its code. The simplest of such transactions is sending Ether from one account to another.
        \item Contract creation call, which creates new accounts with a code associated with it.
    \end{itemize}

    \subsection{Blocks}
    A block is a collection of transactions that are executed in sequence.
    

    \subsection{Ethereum Virtual Machine}
        The formal definition of the EVM is specified in the Ethereum Yellow Paper.~\cite{wood2014ethereum}
        The Ethereum Virtual Machine (EVM) at the heart of the Ethereum blockchain is a VM (virtual machine) with a stack-based architecture with 256-bit word sizes, supporting Turing-complete programming languages.
        EVM handles the computation side for Ethereum and comes with a set of instructions (namely, opcodes).
        Thus, a smart contract, from a low-level point of view, is a series of opcode instructions that EVM can read and compute and execute the logic of that smart contract.
        The EVM is also responsible for estimating and calculating gas consumption for transactions in smart contracts.


\section{Smart Contracts}
    The concept of smart contracts - programs running on the EVM - was first introduced by Nick Szabo in one of his works in 1997.~\cite{szabo1997formalizing}
    They provide a framework that allows any sound program to be executed in an autonomous, distributed, and trusted manner.~\cite{nguyen2020sfuzz}
    The main programming language currently in use for the development of smart contracts is Solidity, although Vyper is gaining gradual traction as well.

    \subsection{Vulnerabilities}
        Solidity, like any other programming language in history, is prone to all kinds of vulnerabilities.
        What makes security vulnerabilities in Solidity so attractive is the fact that the programs written in Solidity are very often used in the financial sector,
        handling millions of dollars in digital assets and cryptocurrencies. Attacking such contracts successfully can result in enormous financial losses.
        Some of these vulnerabilities, like another programming language, arising from the human factor involved in the development of the smart contracts, and some are specific to the blockchain data structures and how they and their components function and interact with each other.
        Furthermore, these are only vulnerabilities within the scope of smart contracts we focus on. Vulnerabilities can arise regarding the blockchains' core infrastructure handling smart contracts.
        In this section, we go over 9 of the more discussed vulnerabilities in Solidity and Ethereum according to ~\cite{dasp} to get a better sense of what threat surface the developers and researchers developing analysis tools face:

            \paragraph{Re-entrancy}
            Often called the most famous Ethereum vulnerability, the reentrancy attack has been a great example of showing the risks of \textit{"Code is Law"} and the importance of smart contract security historically.
            The DAO hack ~\cite{dhillon2017dao} is one of the most famous real-world examples of the reentrancy hack.
            The reentrancy attack can also be counted as a denial-of-service (DoS) attack, where a malicious actor can cause a program to infinitely loop and consume CPU cycles and, in the case of smart contracts, drain a wallet of its ETHs.
            Reentrancy occurs when external contract calls are allowed to make new calls to the calling contract before the initial execution of that call is complete.
            For a function, this means that the contract state may change in the middle of its execution due to a call to an untrusted contract. ~\cite{dasp}

            \paragraph{Access Control}
            The Access Control vulnerability, not exclusive to smart contract types of programs, usually occurs when smart contracts use poor visibility settings regarding calling functions.
            This gives the attackers the ability to try to access the smart contract's private values or hijack the control of the smart contract (for example, becoming the owner of a contract by initializing that contract through a statement like \texttt{owner = msg.sender()}).

            \paragraph{Arithmetic}
            Integer overflows and underflows can cause huge losses in smart contract-based applications~\cite{arithmeticVuln}.
            Values assigned with the integer data type, if not handled carefully concerning being signed or unsigned integers, can cause overflows and underflows and cause DoS-type attacks.
            
            \paragraph{Unhandled Exception}
            Also known as unchecked-send, this vulnerability can cause unwanted outcomes when the smart contract is executed because some low-level calls in Solidity like \texttt{call()} and \texttt{delegatecall()} can return a boolean value set to the value False and lt the execution flow resume if an error happens mid-execution.
            This is not ideal since it means that the execution of the smart contract has not been reversed and successfully completed but with wrong or undesirable outcomes.
            Thus, the return values of such low-level calls should always be checked, and the developers must ensure that such exceptions are handled appropriately during execution.
            
            \paragraph{Frontrunning}
            The frontrunning vulnerability is one of the more famous ones in the list, also known as Transaction Ordering Dependence (TOD).
            Exploiting this vulnerability happens when malicious miners alter the initial default ordering of the transactions submitted to the blockchain.
            Per Eskandari et al.~\cite{eskandari2018frontrunning}, frontrunning can be generally reduced into three templates:
            \begin{itemize}
                \item Displacement attack, where an adversarial party makes a transaction in order to displace the victim user's transaction by having a higher gas price, and thus, the attacker's transaction gets mined before that of the victim's due to it giving having more aligned incentives with he miners' network.
                \item Insertion attack, in which an adversary makes two transactions, one with a higher gas price than the victim transaction and one with a lower gas price, to \textit{sandwich} the victim transaction.~\cite{varun2022mitigating}
                \item Suppression attack, where an attacker makes multiple transactions with higher gas prices than the victim transactions to prevent them from being mined in the same block.
            \end{itemize}

            \paragraph{Bad Randomness}
            Also known as \textit{nothing is secret},~\cite{dasp} this vulnerability happens when smart contracts attempt to generate random, or to be more exact, pseudo-random numbers for any number of reasons.
            If the smart contract generating the pseudo-random number computes that random number using values that a malicious party can guess, then the attacker can predict the next number that will be generated.
            Values such as block timestamps or block numbers are generally advised against being used in such mechanisms. They are called hard-to-predict values, but it is better to use an external oracle to generate the random numbers needed~\cite{swcregistry}.
        
            \paragraph{Time Manipulation}
            This vulnerability is also known as \textit{timestamp dependence}~\cite{dasp}.
            In Solidity, a block's timestamp is often used to generate pseudo-random numbers. In other times, it can be leveraged for smart contracts to conduct time-intensive operations, like unlocking funds at a specific time.
            A malicious miner of a block can manipulate the timestamp reported while generating the block and use this vulnerability for their profit. 
        
            \paragraph{Short Address}
            The short address vulnerability, also known as off-chain issues, results from the Ethereum Virtual Machine accepting arguments with incorrect paddings.
            Attackers exploiting this vulnerability can craft truncated addresses that clients may encode incorrectly in transactions.
            Additionally, it has not been exploited in the wild, as mentioned by ~\cite{ferreira2020smartbugs}.